\chapter*{\centerline{Заключение}}                     % Заголовок
\addcontentsline{toc}{chapter}{Заключение}  % Добавляем его в оглавление

%% Согласно ГОСТ Р 7.0.11-2011:
%% 5.3.3 В заключении диссертации излагают итоги выполненного исследования, рекомендации, перспективы дальнейшей разработки темы.
%% 9.2.3 В заключении автореферата диссертации излагают итоги данного исследования, рекомендации и перспективы дальнейшей разработки темы.
%% Поэтому имеет смысл сделать эту часть общей и загрузить из одного файла в автореферат и в диссертацию:

В процессе работы был реализован децентрализованный мобильный документооборот для вуза. Разработанная система отвечает всем функциональным требованиям, которые были к ней выдвинуты. 

А именно: она позволят создавать и хранить в блокчейне как общие, так и специфические для некоторого типа атрибуты документов. К общим можно отнести название, текст документа и список подписантов, участвующих в жизненном цикле документа. А к специфическим можно отнести факультет, группу и т.д. в случае приказа об отчислении в связи окончанием обучения. Также система позволяет пользователям просматривать список документов в рассмотрении которых они участвуют. Функционал мобильного приложения позволяет настроить отображение списка документов. Такие операции как редактирование, подписание, отклонение, комментирование выполняются пользователями в мобильном приложении с задействованием интерактивности мобильной платформы, а именно инициация данных действий осуществляется <<смахиванием>> документа с представленной пользователю стопки документа. В системе достигнута возможность легкого расширения диапазона шаблонных документов, путем не требующих особых усилий действий разработчика. 
Так же обеспечена последовательность подписи - пользователь, очередь которого одобрить или отклонить по какой либо причине еще не подошла не имеет физической возможности этого сделать. Он может только отслеживать изменение его состояния. Кроме того в разработанном СЭД достигнут достаточный уровень безопасности, за счет того что он основан на технологии блокчейн. Это максимально затрудняет нежелательно изменение состояния хранимых документов.


Тезисы работы были представлены на конференции <<Наука и молодежь 2021>>\cite{my-article} в секции <<Информационные технологии>>.
