% для вертикального центрирования ячеек в tabulary
\def\zz{\ifx\[$\else\aftergroup\zzz\fi}
%$ \] % <-- чиним подсветку синтаксиса в некоторых редакторах
\def\zzz{\setbox0\lastbox
\dimen0\dimexpr\extrarowheight + \ht0-\dp0\relax
\setbox0\hbox{\raise-.5\dimen0\box0}%
\ht0=\dimexpr\ht0+\extrarowheight\relax
\dp0=\dimexpr\dp0+\extrarowheight\relax
\box0
}

\lstdefinelanguage{Go}
{
% Keywords as defined in the language grammar
morekeywords=[1]{%
break,default,func,interface,select,case,defer,go,map,%
struct,chan,else,goto,package,switch,const,fallthrough,%
if,range,type, continue,for,import,return,var},
% Built-in functions
morekeywords=[2]{%
append,cap,close,complex,copy,delete,imag,%
len,make,new,panic,print,println,real,recover},
% Pre-declared types
morekeywords=[3]{%
bool,byte,complex64,complex128,error,float32,float64,%
int,int8,int16,int32,int64,rune,string,%
uint,uint8,uint16,uint32,uint64,uintptr},
% Constants and zero value
morekeywords=[4]{true,false,iota,nil},
% Strings : "foo", 'bar', `baz`
morestring=[b]{"},
morestring=[b]{'},
morestring=[b]{`},
% Comments : /* comment */ and // comment
comment=[l]{//},
morecomment=[s]{/*}{*/},
% Options
sensitive=true
}
\lstdefinelanguage{JavaScript}{
alsoletter={.},
keywords={arguments,await,break,case,catch,class,const,continue,debugger,default,delete,do,else,enum,eval,export,extends,false,finally,for,function,if,implements,import,in,instanceof,interface,let,new,null,package,private,protected,public,return,static,super,switch,this,throw,true,try,typeof,var,void,while,with,yield}, % JavaScript ES6 keywords
ndkeywords={add, apply, args, Array, Array.from, Array.isArray, Array.of , Array.prototype, ArrayBuffer, bind, Boolean, call, charAt, charCodeAt, clear, codePointAt, concat, constructor, copyWithin, DataView, Date, Date.now, Date.parse, Date.prototype, Date.UTC, decodeURI, decodeURIComponent, encodeURI, encodeURIComponent, endsWith, entries, Error, Error.prototype, EvalError, every, false, fill, filter, find, findIndex, Float32Array, Float64Array, forEach, FulfillPromise, Function, Function.length, get, getDate, getDay, getFullYear, getHours, getMilliseconds, getMinutes, getMonth, getSeconds, getTime, getTimezoneOffset, getUTCDate, getUTCDay, getUTCFullYear, getUTCHours, getUTCMilliseconds, getUTCMinutes, getUTCMonth, getUTCSeconds, has,hasInstance, hasOwnProperty, ignoreCase, includes, indexOf, indexOf, Infinity, Int8Array, Int16Array, Int32Array, isConcatSpreadable, isFinite, isNaN, IsPromise, isPrototypeOf, Iterable, iterator, join, JSON, JSON.parse, JSON.stringify, keys, lastIndexOf, lastIndexOf, length, localeCompare, map, Map, match, match, Math, Math.abs , Math.acos, Math.acosh, Math.asin, Math.asinh, Math.atan, Math.atan2, Math.atanh, Math.cbrt, Math.ceil, Math.clz32, Math.cos, Math.cosh, Math.E, Math.exp, Math.expm1, Math.floor, Math.fround, Math.hypot, Math.imul, Math.LN2, Math.LN10, Math.log, Math.log1p, Math.log2, Math.LOG2E, Math.log10, Math.LOG10E, Math.max, Math.min, Math.PI, Math.pow, Math.random, Math.round, Math.sign, Math.sin, Math.sinh, Math.sqrt, Math.SQRT1_2, Math.SQRT2, Math.tan, Math.tanh, Math.trunc, message, multiline, name, NaN, NewPromiseCapability, next, normalize, null, Number, Number.EPSILON, Number.isFinite, Number.isInteger, Number.isNaN, Number.isSafeInteger, Number.MAX_SAFE_INTEGER, Number.MAX_VALUE, Number.MIN_SAFE_INTEGER, Number.MIN_VALUE, Number.NaN, Number.NEGATIVE_INFINITY, Number.parseFloat, Number.parseInt, Number.POSITIVE_INFINITY, Number.prototype, Object, Object, Object.assign, Object.create, Object.defineProperties, Object.defineProperty, Object.freeze, Object.getOwnPropertyDescriptor, Object.getOwnPropertyNames, Object.getOwnPropertySymbols, Object.getPrototypeOf, Object.is, Object.isExtensible, Object.isFrozen, Object.isSealed, Object.keys, Object.preventExtensions, Object.prototype, Object.seal, Object.setPrototypeOf, of, parseFloat, parseInt, pop, Promise, Promise.all , Promise.race, Promise.reject, Promise.resolve, PromiseReactionJob, propertyIsEnumerable, prototype, Proxy, Proxy.revocable , push, RangeError, reduce, reduceRight, ReferenceError, Reflect, Reflect.apply, Reflect.construct , Reflect.defineProperty, Reflect.deleteProperty, Reflect.enumerate, Reflect.get, Reflect.getOwnPropertyDescriptor, Reflect.getPrototypeOf, Reflect.has, Reflect.isExtensible, Reflect.ownKeys, Reflect.preventExtensions, Reflect.set, Reflect.setPrototypeOf, Reflection, RegExp, RegExp, RegExp.prototype, repeat, replace, replace, reverse, search, search, Set, set, setDate, setFullYear, setHours, setMilliseconds, setMinutes, setMonth, setSeconds, setTime, setUTCDate, setUTCFullYear, setUTCHours, setUTCMilliseconds, setUTCMinutes, setUTCMonth, setUTCSeconds, shift, slice, slice, some, sort, species, splice, split, split, startsWith, String, String.fromCharCode, String.fromCodePoint, String.raw, substring, Symbol, Symbol.for, Symbol.hasInstance, Symbol.isConcatSpreadable, Symbol.iterator, Symbol.keyFor, Symbol.match, Symbol.prototype, Symbol.replace, Symbol.replace, Symbol.search, Symbol.species, Symbol.split, Symbol.toPrimitive, Symbol.toStringTag, Symbol.unscopables, SyntaxError, then, toDateString, toExponential, toFixed, toISOString, toJSON, toLocaleDateString, toLocaleLowerCase, toLocaleString, toLocaleString, toLocaleString, toLocaleString, toLocaleTimeString, toLocaleUpperCase, toLowerCase, toPrecision, toPrimitive, toString, toStringTag, toTimeString, toUpperCase, toUTCString, TriggerPromiseReactions, trim, true, TypeError, Uint8Array, Uint8ClampedArray, Uint16Array, Uint32Array, undefined, unscopables, unshift, URIError, valueOf, WeakMap, WeakSet
}, % JavaScript extended keywords
sensitive=true,
morestring=[b]",
morestring=[d]',
comment=[l]{//},
morecomment=[s]{/*}{*/}
}

%решаем проблему с кириллицей в комментариях (в pdflatex) https://tex.stackexchange.com/a/103712
\lstset{extendedchars=true,keepspaces=true,literate={Ö}{{\"O}}1
{Ä}{{\"A}}1
{Ü}{{\"U}}1
{ß}{{\ss}}1
{ü}{{\"u}}1
{ä}{{\"a}}1
{ö}{{\"o}}1
{~}{{\textasciitilde}}1
{а}{{\selectfont\char224}}1
{б}{{\selectfont\char225}}1
{в}{{\selectfont\char226}}1
{г}{{\selectfont\char227}}1
{д}{{\selectfont\char228}}1
{е}{{\selectfont\char229}}1
{ё}{{\"e}}1
{ж}{{\selectfont\char230}}1
{з}{{\selectfont\char231}}1
{и}{{\selectfont\char232}}1
{й}{{\selectfont\char233}}1
{к}{{\selectfont\char234}}1
{л}{{\selectfont\char235}}1
{м}{{\selectfont\char236}}1
{н}{{\selectfont\char237}}1
{о}{{\selectfont\char238}}1
{п}{{\selectfont\char239}}1
{р}{{\selectfont\char240}}1
{с}{{\selectfont\char241}}1
{т}{{\selectfont\char242}}1
{у}{{\selectfont\char243}}1
{ф}{{\selectfont\char244}}1
{х}{{\selectfont\char245}}1
{ц}{{\selectfont\char246}}1
{ч}{{\selectfont\char247}}1
{ш}{{\selectfont\char248}}1
{щ}{{\selectfont\char249}}1
{ъ}{{\selectfont\char250}}1
{ы}{{\selectfont\char251}}1
{ь}{{\selectfont\char252}}1
{э}{{\selectfont\char253}}1
{ю}{{\selectfont\char254}}1
{я}{{\selectfont\char255}}1
{А}{{\selectfont\char192}}1
{Б}{{\selectfont\char193}}1
{В}{{\selectfont\char194}}1
{Г}{{\selectfont\char195}}1
{Д}{{\selectfont\char196}}1
{Е}{{\selectfont\char197}}1
{Ё}{{\"E}}1
{Ж}{{\selectfont\char198}}1
{З}{{\selectfont\char199}}1
{И}{{\selectfont\char200}}1
{Й}{{\selectfont\char201}}1
{К}{{\selectfont\char202}}1
{Л}{{\selectfont\char203}}1
{М}{{\selectfont\char204}}1
{Н}{{\selectfont\char205}}1
{О}{{\selectfont\char206}}1
{П}{{\selectfont\char207}}1
{Р}{{\selectfont\char208}}1
{С}{{\selectfont\char209}}1
{Т}{{\selectfont\char210}}1
{У}{{\selectfont\char211}}1
{Ф}{{\selectfont\char212}}1
{Х}{{\selectfont\char213}}1
{Ц}{{\selectfont\char214}}1
{Ч}{{\selectfont\char215}}1
{Ш}{{\selectfont\char216}}1
{Щ}{{\selectfont\char217}}1
{Ъ}{{\selectfont\char218}}1
{Ы}{{\selectfont\char219}}1
{Ь}{{\selectfont\char220}}1
{Э}{{\selectfont\char221}}1
{Ю}{{\selectfont\char222}}1
{Я}{{\selectfont\char223}}1
{і}{{\selectfont\char105}}1
{ї}{{\selectfont\char168}}1
{є}{{\selectfont\char185}}1
{ґ}{{\selectfont\char160}}1
{І}{{\selectfont\char73}}1
{Ї}{{\selectfont\char136}}1
{Є}{{\selectfont\char153}}1
{Ґ}{{\selectfont\char128}}1
}

% Ширина текста минус ширина надписи 999
\newlength{\twless}
\newlength{\lmarg}
\setlength{\lmarg}{\widthof{999}}   % ширина надписи 999
\setlength{\twless}{\textwidth-\lmarg}

\definecolor{mediumgray}{rgb}{0.3, 0.4, 0.4}
\definecolor{mediumblue}{rgb}{0.0, 0.0, 0.8}
\definecolor{forestgreen}{rgb}{0.13, 0.55, 0.13}
\definecolor{darkviolet}{rgb}{0.58, 0.0, 0.83}
\definecolor{royalblue}{rgb}{0.25, 0.41, 0.88}
\definecolor{crimson}{rgb}{0.86, 0.8, 0.24}

\lstset{
backgroundcolor=\color{white},  % choose the background color. You must add \usepackage{color}
showspaces=false,               % show spaces adding particular underscores
showstringspaces=false,         % underline spaces within strings
showtabs=false,                 % show tabs within strings adding particular underscores
%frame=leftline,                 % adds a frame of different types around the code
rulecolor=\color{black},        % if not set, the frame-color may be changed on line-breaks within not-black text (e.g. commens (green here))
tabsize=2,                      % sets default tabsize to 2 spaces
captionpos=t,                   % sets the caption-position to top
breaklines=true,                % sets automatic line breaking
breakatwhitespace=false,        % sets if automatic breaks should only happen at whitespace
basicstyle=\fontsize{12pt}{14pt}\selectfont\ttfamily,% the size of the fonts that are used for the code
keywordstyle=\color{mediumblue},
keywordstyle={[2]\color{darkviolet}},
keywordstyle={[3]\color{royalblue}},
commentstyle=\color{mediumgray}\upshape,
emphstyle=\color{crimson},
rulecolor=\color{black},
numberstyle=\tiny\color{black},
stringstyle=\color{forestgreen},   % string literal style
escapeinside={\%*}{*)},         % if you want to add a comment within your code
morekeywords={*,...},           % if you want to add more keywords to the set
inputencoding=utf8,             % кодировка кода
xleftmargin={\lmarg},           % Чтобы весь код и полоска с номерами строк была смещена влево, так чтобы цифры не вылезали за пределы текста слева
}

%http://tex.stackexchange.com/questions/26872/smaller-frame-with-listings
% Окружение, чтобы листинг был компактнее обведен рамкой, если она задается, а не на всю ширину текста
\makeatletter
\newenvironment{SmallListing}[1][]
{\lstset{#1}\VerbatimEnvironment\begin{VerbatimOut}{VerbEnv.tmp}}
{\end{VerbatimOut}\settowidth\@tempdima{%
\lstinputlisting{VerbEnv.tmp}}
\minipage{\@tempdima}\lstinputlisting{VerbEnv.tmp}\endminipage}
\makeatother

\DefineVerbatimEnvironment% с шрифтом 12 пт
{Verb}{Verbatim}
{fontsize=\fontsize{12pt}{14pt}\selectfont}

\newfloat[chapter]{ListingEnv}{lol}{Листинг}

\renewcommand{\lstlistingname}{Листинг}

%Общие счётчики окружений листингов
%http://tex.stackexchange.com/questions/145546/how-to-make-figure-and-listing-share-their-counter
% Если смешивать плавающие и не плавающие окружения, то могут быть проблемы с нумерацией
\makeatletter
\AtBeginDocument{%
\let\c@ListingEnv\c@lstlisting
\let\theListingEnv\thelstlisting
\let\ftype@lstlisting\ftype@ListingEnv % give the floats the same precedence
}
\makeatother

% значок С++ — используйте команду \cpp
\newcommand{\cpp}{%
C\nolinebreak\hspace{-.05em}%
\raisebox{.2ex}{+}\nolinebreak\hspace{-.10em}%
\raisebox{.2ex}{+}%
}

%%%  Чересстрочное форматирование таблиц
%% http://tex.stackexchange.com/questions/278362/apply-italic-formatting-to-every-other-row
\newcounter{rowcnt}
\newcommand\altshape{\ifnumodd{\value{rowcnt}}{\color{red}}{\vspace*{-1ex}\itshape}}
% \AtBeginEnvironment{tabular}{\setcounter{rowcnt}{1}}
% \AtEndEnvironment{tabular}{\setcounter{rowcnt}{0}}

%%% Ради примера во второй главе
\let\originalepsilon\epsilon
\let\originalphi\phi
\let\originalkappa\kappa
\let\originalle\le
\let\originalleq\leq
\let\originalge\ge
\let\originalgeq\geq
\let\originalemptyset\emptyset
\let\originaltan\tan
\let\originalcot\cot
\let\originalcsc\csc

%%% Русская традиция начертания математических знаков
\renewcommand{\le}{\ensuremath{\leqslant}}
\renewcommand{\leq}{\ensuremath{\leqslant}}
\renewcommand{\ge}{\ensuremath{\geqslant}}
\renewcommand{\geq}{\ensuremath{\geqslant}}
\renewcommand{\emptyset}{\varnothing}

%%% Русская традиция начертания математических функций (на случай копирования из зарубежных источников)
\renewcommand{\tan}{\operatorname{tg}}
\renewcommand{\cot}{\operatorname{ctg}}
\renewcommand{\csc}{\operatorname{cosec}}

%%% Русская традиция начертания греческих букв (греческие буквы вертикальные, через пакет upgreek)
\renewcommand{\epsilon}{\ensuremath{\upvarepsilon}}   %  русская традиция записи
\renewcommand{\phi}{\ensuremath{\upvarphi}}
%\renewcommand{\kappa}{\ensuremath{\varkappa}}
\renewcommand{\alpha}{\upalpha}
\renewcommand{\beta}{\upbeta}
\renewcommand{\gamma}{\upgamma}
\renewcommand{\delta}{\updelta}
\renewcommand{\varepsilon}{\upvarepsilon}
\renewcommand{\zeta}{\upzeta}
\renewcommand{\eta}{\upeta}
\renewcommand{\theta}{\uptheta}
\renewcommand{\vartheta}{\upvartheta}
\renewcommand{\iota}{\upiota}
\renewcommand{\kappa}{\upkappa}
\renewcommand{\lambda}{\uplambda}
\renewcommand{\mu}{\upmu}
\renewcommand{\nu}{\upnu}
\renewcommand{\xi}{\upxi}
\renewcommand{\pi}{\uppi}
\renewcommand{\varpi}{\upvarpi}
\renewcommand{\rho}{\uprho}
%\renewcommand{\varrho}{\upvarrho}
\renewcommand{\sigma}{\upsigma}
%\renewcommand{\varsigma}{\upvarsigma}
\renewcommand{\tau}{\uptau}
\renewcommand{\upsilon}{\upupsilon}
\renewcommand{\varphi}{\upvarphi}
\renewcommand{\chi}{\upchi}
\renewcommand{\psi}{\uppsi}
\renewcommand{\omega}{\upomega}

\def\slantfrac#1#2{ \hspace{3pt}\!^{#1}\!\!\hspace{1pt}/
\hspace{2pt}\!\!_{#2}\!\hspace{3pt}
} %Макрос для красивых дробей в строчку (например, 1/2)
