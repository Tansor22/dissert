
\section*{Общая характеристика работы}

\newcommand{\actuality}{\underline{\textbf{\actualityTXT}}}
\newcommand{\progress}{\underline{\textbf{\progressTXT}}}
\newcommand{\aim}{\underline{{\textbf\aimTXT}}}
\newcommand{\tasks}{\underline{\textbf{\tasksTXT}}}
\newcommand{\novelty}{\underline{\textbf{\noveltyTXT}}}
\newcommand{\influence}{\underline{\textbf{\influenceTXT}}}
\newcommand{\methods}{\underline{\textbf{\methodsTXT}}}
\newcommand{\defpositions}{\underline{\textbf{\defpositionsTXT}}}
\newcommand{\reliability}{\underline{\textbf{\reliabilityTXT}}}
\newcommand{\probation}{\underline{\textbf{\probationTXT}}}
\newcommand{\contribution}{\underline{\textbf{\contributionTXT}}}
\newcommand{\publications}{\underline{\textbf{\publicationsTXT}}}


{\actuality} 
Часть обывателей, под словом blockcahin понимает криптовалюты, биржевые рынки и прочие элементы криптографической коммерции.
Другая же, более искушенная часть, считает что блокчейн это не что иное как большие криптографические связные списки. Ничто из этого нельзя назвать неправдой, а также невозможно не отметить растущий интерес к этой технологии в последнее время. Технология блокчейн, на которую опирается великое множество криптовалют может быть использована в гораздо больших областях человеческой деятельности, чем это может представить себе обыватель.
Например, на основе этой технологии создаются невзаимозаменяемые токены (англ. non-fungible token, NFT) – особый вид криптографеских токенов, любой из которых уникален и является единственным в мире. Такие токены используются на некоторых площадках для подтверждения факта владения чем либо, как правило уникальными цифровыми ценностями, такими как предметы цифрового искусства. Таким образом, люди приобретают цифровые ценности, которые становятся их имуществом. Словом, данная технологи широко востребована множеством различных организаций. Многие компании постепенно внедряют её в свою инфраструктуру, при этом используя как открытые публичные блокчейн платформы, так и платформы enterprise-уровня. С использованием корпоративных [enterprise] блокчейн-решений компании также автоматизируют некоторые бизнес-процессы, реализуя кейсы использования блокчейн в программных продуктах.

Говоря же об инструментах enterprise-уровня, следует отметить, что самые популярные средства разработки частных блокчейн-систем представлены фреймворками консорциума Hyperledger. Крупные компании обращаются к технологии блокчейн, когда им необходима система проведения транзакций для различных бизнес процессов с повышенными требованиями к безопасности данных. Одной из областей, в которой требуется высокий уровень безопасности является область документооборота. А в этой области цифровизация также весьма перспективна - электронный документооборот - это то, что желают внедрить в свою инфраструктуру великое множество организаций. Таким образом, создание системы электронного документооборота видится очень перспективным и полезным. Об этом и пойдет речь на страницах данной работы, а также будет рассмотрен фреймворк Hyperledger Fabric, как инструмент создания приватных сетей блокчейна.

 % Характеристика работы по структуре во введении и в автореферате не отличается (ГОСТ Р 7.0.11, пункты 5.3.1 и 9.2.1), потому её загружаем из одного и того же внешнего файла, предварительно задав форму выделения некоторым параметрам

%Диссертационная работа была выполнена при поддержке грантов ...

%\underline{\textbf{Объем и структура работы.}} Диссертация состоит из~введения,
%четырех глав, заключения и~приложения. Полный объем диссертации
%\textbf{ХХХ}~страниц текста с~\textbf{ХХ}~рисунками и~5~таблицами. Список
%литературы содержит \textbf{ХХX}~наименование.

\section*{Содержание работы}
Во \underline{\textbf{введении}} обосновывается актуальность
исследований, проводимых в~рамках данной диссертационной работы,
приводится обзор научной литературы по изучаемой проблеме,
формулируется цель, ставятся задачи работы, излагается научная новизна
и практическая значимость представляемой работы. В~последующих главах
сначала описывается общий принцип, позволяющий ..., а~потом идёт
апробация на частных примерах: ...  и~... .


\underline{\textbf{Первая глава}} посвящена ...

 картинку можно добавить так:
\begin{figure}[ht]
  \centering
  \includegraphics [scale=0.27] {latex}
  \caption{Подпись к картинке.}
  \label{fig:latex}
\end{figure}

Формулы в строку без номера добавляются так:
\[
  \lambda_{T_s} = K_x\frac{d{x}}{d{T_s}}, \qquad
  \lambda_{q_s} = K_x\frac{d{x}}{d{q_s}},
\]

\underline{\textbf{Вторая глава}} посвящена исследованию

\underline{\textbf{Третья глава}} посвящена исследованию

Можно сослаться на свои работы в автореферате. Для этого в файле
\verb!Synopsis/setup.tex! необходимо присвоить положительное значение
счётчику \verb!\setcounter{usefootcite}{1}!. В таком случае ссылки на
работы других авторов будут подстрочными.
\ifnumgreater{\value{usefootcite}}{0}{
Изложенные в третьей главе результаты опубликованы в~\cite{vakbib1, vakbib2}.
}{}
Использование подстрочных ссылок внутри таблиц может вызывать проблемы.

В \underline{\textbf{четвертой главе}} приведено описание

В \underline{\textbf{заключении}} приведены основные результаты работы, которые заключаются в следующем:
%% Согласно ГОСТ Р 7.0.11-2011:
%% 5.3.3 В заключении диссертации излагают итоги выполненного исследования, рекомендации, перспективы дальнейшей разработки темы.
%% 9.2.3 В заключении автореферата диссертации излагают итоги данного исследования, рекомендации и перспективы дальнейшей разработки темы.



При использовании пакета \verb!biblatex! список публикаций автора по теме
диссертации формируется в разделе <<\publications>>\ файла
\verb!common/characteristic.tex!  при помощи команды \verb!\nocite!

\ifdefmacro{\microtypesetup}{\microtypesetup{protrusion=false}}{} % не рекомендуется применять пакет микротипографики к автоматически генерируемому списку литературы
\ifnumequal{\value{bibliosel}}{0}{% Встроенная реализация с загрузкой файла через движок bibtex8
  \renewcommand{\bibname}{\large \bibtitleauthor}
  \nocite{*}
  \insertbiblioauthor           % Подключаем Bib-базы
  %\insertbiblioexternal   % !!! bibtex не умеет работать с несколькими библиографиями !!!
}{% Реализация пакетом biblatex через движок biber
  \ifnumgreater{\value{usefootcite}}{0}{
%  \nocite{*} % Невидимая цитата всех работ, позволит вывести все работы автора
  \insertbiblioauthorcited      % Вывод процитированных в автореферате работ автора
  }{
  \insertbiblioauthor           % Вывод всех работ автора
%  \insertbiblioauthorgrouped    % Вывод всех работ автора, сгруппированных по источникам
%  \insertbiblioauthorimportant  % Вывод наиболее значимых работ автора (определяется в файле characteristic во второй section)
  \insertbiblioexternal            % Вывод списка литературы, на которую ссылались в тексте автореферата
  }
}
\ifdefmacro{\microtypesetup}{\microtypesetup{protrusion=true}}{}
