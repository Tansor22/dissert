
{\actuality} 
Часть обывателей, под словом blockcahin понимает криптовалюты, биржевые рынки и прочие элементы криптографической коммерции.
Другая же, более искушенная часть, считает что блокчейн это не что иное как большие криптографические связные списки. Ничто из этого нельзя назвать неправдой, а также невозможно не отметить растущий интерес к этой технологии в последнее время. Технология блокчейн, на которую опирается великое множество криптовалют может быть использована в гораздо больших областях человеческой деятельности, чем это может представить себе обыватель.
Например, на основе этой технологии создаются невзаимозаменяемые токены\cite{nft} (англ. non-fungible token, NFT) – особый вид криптографеских токенов, любой из которых уникален и является единственным в мире. Такие токены используются на некоторых площадках для подтверждения факта владения чем либо, как правило уникальными цифровыми ценностями, такими как предметы цифрового искусства. Таким образом, люди приобретают цифровые ценности, которые становятся их имуществом. Словом, данная технологи широко востребована множеством различных организаций. Многие компании постепенно внедряют её в свою инфраструктуру, при этом используя как открытые публичные блокчейн платформы, так и платформы enterprise-уровня. С использованием корпоративных блокчейн-решений компании также автоматизируют некоторые бизнес-процессы, реализуя кейсы использования блокчейн в программных продуктах.

Говоря же об инструментах enterprise-уровня, следует отметить, что самые популярные средства разработки частных блокчейн-систем представлены фреймворками консорциума Hyperledger. Крупные компании обращаются к технологии блокчейн, когда им необходима система проведения транзакций для различных бизнес процессов с повышенными требованиями к безопасности данных. Одной из областей, в которой требуется высокий уровень безопасности является область документооборота. А в этой области цифровизация также весьма перспективна - электронный документооборот - это то, что желают внедрить в свою инфраструктуру великое множество организаций. Таким образом, создание системы электронного документооборота видится очень перспективным и полезным. Об этом и пойдет речь на страницах данной работы, а также будет рассмотрен фреймворк Hyperledger Fabric, как инструмент создания приватных сетей блокчейна.

