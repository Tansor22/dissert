%% Согласно ГОСТ Р 7.0.11-2011:
%% 5.3.3 В заключении диссертации излагают итоги выполненного исследования, рекомендации, перспективы дальнейшей разработки темы.
%% 9.2.3 В заключении автореферата диссертации излагают итоги данного исследования, рекомендации и перспективы дальнейшей разработки темы.
\begin{enumerate}
  \item В процессе выполнения работы была подробно исследована предметная область задачи. Были рассмотрены некоторые предлагаемые решения относительно тестирования как открытых блокчейн-экосистем, так и приватных, описаны проблемы тестирования блокчейн-приложений в целом.
  \item Был проведен обзор методик DevOps и особенностей их внедрения, а так же их влияния на качество выпускаемого программного обеспечения. Выяснилось, что популярные экосистемы применяют некоторые методики для повышения качества своих продуктов.
  \item Благодаря обзору имеющихся инструментов, поставлена задача на внедрения некоторых DevOps процессов относительно разработки блокчейн-приложений. Проведенный обзор позволил получить понимание архитектуры тестового стенда.
  \item Разработанная архитектура тестовых стендов для криптовалютных платежных шлюзов успешно показала себя в некоторых реальных проектах. Она была применена не только для тестирования платежных шлюзов, но в том числе и для тестирования HTTP функций обратного вызова различных сервисов.
  \item Итоговый тестовый стенд позволил обнаружить и устранить некоторые серьезные ошибки в приложениях, а также выявил уязвимости приложений при различном поведении блокчейн-сетей, не определяемые ручным тестированием и которые впоследствии были проработаны и устранены.
  \item Подход с использованием контейнерной виртуализации для компонентов Hyperledger Sawtooth позволил внедрить в разработку на данном фреймворке некоторые DevOps-методики, которые впоследствии положительно повлияли на качество продукта.
  \item Компоненты обеих инфраструктур нашли свое место в реальных продуктах, системах.
\end{enumerate}
