% Общие стили оформления.
% Возможные варианты значений ищите в описании библиотеки beamer
\usetheme{Pittsburgh}
\usecolortheme{whale}
% Listings
\definecolor{mediumgray}{rgb}{0.3, 0.4, 0.4}
\definecolor{mediumblue}{rgb}{0.0, 0.0, 0.8}
\definecolor{forestgreen}{rgb}{0.13, 0.55, 0.13}
\definecolor{darkviolet}{rgb}{0.58, 0.0, 0.83}
\definecolor{royalblue}{rgb}{0.25, 0.41, 0.88}
\definecolor{crimson}{rgb}{0.86, 0.8, 0.24}

\lstset{
backgroundcolor=\color{white},  % choose the background color. You must add \usepackage{color}
showspaces=false,               % show spaces adding particular underscores
showstringspaces=false,         % underline spaces within strings
showtabs=false,                 % show tabs within strings adding particular underscores
%frame=leftline,                 % adds a frame of different types around the code
rulecolor=\color{black},        % if not set, the frame-color may be changed on line-breaks within not-black text (e.g. commens (green here))
tabsize=2,                      % sets default tabsize to 2 spaces
captionpos=t,                   % sets the caption-position to top
breaklines=true,                % sets automatic line breaking
breakatwhitespace=false,        % sets if automatic breaks should only happen at whitespace
basicstyle=\fontsize{12pt}{14pt}\selectfont\ttfamily,% the size of the fonts that are used for the code
keywordstyle=\color{mediumblue},
keywordstyle={[2]\color{darkviolet}},
keywordstyle={[3]\color{royalblue}},
commentstyle=\color{mediumgray}\upshape,
emphstyle=\color{crimson},
rulecolor=\color{black},
numberstyle=\tiny\color{black},
stringstyle=\color{forestgreen},   % string literal style
escapeinside={\%*}{*)},         % if you want to add a comment within your code
morekeywords={*,...},           % if you want to add more keywords to the set
inputencoding=utf8,             % кодировка кода
xleftmargin={\lmarg},           % Чтобы весь код и полоска с номерами строк была смещена влево, так чтобы цифры не вылезали за пределы текста слева
}

% \usetheme[secheader]{Boadilla}
% \usecolortheme{seahorse}

% выключение кнопок навигации
\beamertemplatenavigationsymbolsempty

% Размеры шрифтов
\setbeamerfont{title}{size=\large}
\setbeamerfont{subtitle}{size=\small}
\setbeamerfont{author}{size=\normalsize}
\setbeamerfont{institute}{size=\small}
\setbeamerfont{date}{size=\normalsize}
\setbeamerfont{bibliography item}{size=\small}
\setbeamerfont{bibliography entry author}{size=\small}
\setbeamerfont{bibliography entry title}{size=\small}
\setbeamerfont{bibliography entry location}{size=\small}
\setbeamerfont{bibliography entry note}{size=\small}
% Аналогично можно настроить и другие размеры.
% Названия классов элементов можно найти здесь
% http://www.cpt.univ-mrs.fr/~masson/latex/Beamer-appearance-cheat-sheet.pdf

% Цвет элементов
\setbeamercolor{footline}{fg=blue}
\setbeamercolor{bibliography item}{fg=black}
\setbeamercolor{bibliography entry author}{fg=black}
\setbeamercolor{bibliography entry title}{fg=black}
\setbeamercolor{bibliography entry location}{fg=black}
\setbeamercolor{bibliography entry note}{fg=black}
% Аналогично можно настроить и другие цвета.
% Названия классов элементов можно найти здесь
% http://www.cpt.univ-mrs.fr/~masson/latex/Beamer-appearance-cheat-sheet.pdf

% Убрать иконки перед списком литературы
% https://tex.stackexchange.com/a/124271/104425
\setbeamertemplate{bibliography item}{}

% Использовать шрифт с засечками для формул
% https://tex.stackexchange.com/a/34267/104425
\usefonttheme[onlymath]{serif}

% https://tex.stackexchange.com/a/291545/104425
\makeatletter
\def\beamer@framenotesbegin{% at beginning of slide
    \usebeamercolor[fg]{normal text}
    \gdef\beamer@noteitems{}%
    \gdef\beamer@notes{}%
}
\makeatother

% footer презентации
\setbeamertemplate{footline}{
    \leavevmode%
    \hbox{%
        \begin{beamercolorbox}[wd=.333333\paperwidth,ht=2.25ex,dp=1ex,center]{}%
            % И. О. Фамилия, Организация кратко
            \thesisAuthorShort, \thesisOrganizationShort
        \end{beamercolorbox}%
        \begin{beamercolorbox}[wd=.333333\paperwidth,ht=2.25ex,dp=1ex,center]{}%
            % Город, 20XX
            \thesisCity, \thesisYear
        \end{beamercolorbox}%
        \begin{beamercolorbox}[wd=.333333\paperwidth,ht=2.25ex,dp=1ex,right]{}%
            Стр. \insertframenumber{} из \inserttotalframenumber \hspace*{2ex}
        \end{beamercolorbox}}%
    \vskip0pt%
}

% вывод на экран заметок к презентации
\ifnumequal{\value{presnotes}}{0}{}{
    \setbeameroption{show notes}
    \ifnumequal{\value{presnotes}}{2}{
        \setbeameroption{show notes on second screen=\presposition}
    }{}
}
